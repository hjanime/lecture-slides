% https://tex.stackexchange.com/questions/343851/chemfig-submolecule-that-takes-an-argument

\usepackage{chemfig}
\usepackage{xstring}

\makeatletter

% he sent me this later ... it seems this is simply a slight refactoring.

\newcommand*\if@csfirst[1]{%
    \csname @\ifcat\relax\expandafter\noexpand\@car#1\@nil first\else second\fi oftwo\endcsname
}

\newcommand*\derivesubmol[4]{%
    \saveexpandmode\saveexploremode\expandarg\exploregroups
    \if@csfirst{#2}
        {\expandafter\StrSubstitute\@car#2\@nil}
        {\expandafter\StrSubstitute\csname CF@@#2\endcsname}
    {\@empty#3}{\@empty#4}[\temp@]%
    \if@csfirst{#1}
        {\expandafter\let\@car#1\@nil}
        {\expandafter\let\csname CF@@#1\endcsname}\temp@
    \restoreexpandmode\restoreexploremode
}

\newcommand*\showsubmol[1]{%
    \if@csfirst{#1}%
        {\begingroup submol "\expandafter\showsubmol@i\string#1" = \ttfamily
        \expandafter\expandafter\expandafter\def\expandafter\expandafter\expandafter#1%
            \expandafter\expandafter\expandafter{\expandafter\@gobble#1}%
        \expandafter\expandafter\expandafter\strip@prefix\expandafter\meaning\@car#1\@nil
        \endgroup}%
        {\expandafter\showsubmol\csname CF@@#1\endcsname}%
}

\def\showsubmol@i#1#2#3#4#5{}

\newcommand*\exp@addtomacro[2]{\expandafter\@xs@addtomacro\expandafter#1\expandafter{#2}}

\newcommand*\expandsubmol[1]{%
    \if@csfirst{#1}%
        {\saveexpandmode\saveexploremode\expandarg\noexploregroups
        \let\parsed@mol\@empty\let\remain@mol#1%
        \IfSubStr#1!%
            {\expandsubmol@i
            \let#1\parsed@mol
            }%
            \relax
        }%
        {\expandafter\expandsubmol\csname CF@@#1\endcsname}%
}

\newcommand*\expandsubmol@i{%
    \StrBefore\remain@mol![\temp@]%
    \exp@addtomacro\parsed@mol\temp@
    \StrBehind\remain@mol![\remain@mol]%
    \StrSplit\remain@mol\@ne\remain@mol\temp@
    \StrRemoveBraces\remain@mol[\remain@mol]%
    \expandafter\if@csfirst\expandafter{\remain@mol}%
        {\expandafter\let\expandafter\remain@mol\remain@mol}%
        {\expandafter\let\expandafter\remain@mol\csname CF@@\remain@mol\endcsname}%
    \StrGobbleLeft\remain@mol\@ne[\remain@mol]%
    \exp@addtomacro\remain@mol\temp@
    \IfSubStr\remain@mol!%
        \expandsubmol@i
        {\exp@addtomacro\parsed@mol\remain@mol
        \restoreexpandmode\restoreexploremode
        }%
}

\makeatother

% cosmetic enhancements for the example
\setcrambond{1.75pt}{0.4pt}{1.0pt} % previously too crammed for print.
\setatomsep{18pt}
\renewcommand*\printatom[1]{\ensuremath{\mathsf{#1}}}

\tikzset{ % bond in the foreground
    fgbond/.style={% foreground bond - connecting two cram bonds.
        line width=1.6pt,
        shorten <=-.6pt,
        shorten >=-.6pt
   }
}

% define named substituent dummies. Not strictly necessary, but useful for
% visual display of the template that contains them.
\definesubmol{rt1}{-[:90]rt1}
\definesubmol{rt2}{-[:-90,0.6]rt2}

% define a template that contains named dummy substituents that can be replaced
\definesubmol{ribosetemplate}{%
        (
            -[:28,1.508]O
            -[:-28,1.508]
        )
    <[:-45]
        (
          -[0,1.25,,,fgbond]
              (!{rt2})
          >[:45]
              (!{rt1})
         )
}

% partially specialize the template by supplying a substituent for the ribose
\derivesubmol{ribonucleoside}{ribosetemplate}{!{rt2}}{-[6,0.9]OH}
\derivesubmol{deoxyribonucleoside}{ribosetemplate}{!{rt2}}{}

% define some more submols
\definesubmol{adenine}{N*5([::-18]-*6(-N=-N=(-NH_2)-)=-N=-)}
\redefinesubmol{guanine}{N*5([::-18]-*6(-N=(-NH_2)-NH-(=O)-)=-N=-)}

% put those into the second position
\derivesubmol{guanosine}{ribonucleoside}{!{rt1}}{-[2]!{guanine}}
\derivesubmol{deoxyadenosine}{deoxyribonucleoside}{!{rt1}}{-[2]!{adenine}}
